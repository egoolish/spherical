\documentclass[12pt, leqno]{article}
%\usepackage{fullpage}
\usepackage{epic}
\usepackage{eepic}
\usepackage{paralist}
\usepackage{graphicx}
\usepackage{algorithm,algorithmic}
\usepackage{tikz}
\usepackage{xcolor,colortbl}
\usepackage{wrapfig}
\usepackage{amsthm}
\usepackage{amsmath}
\usepackage{listings}
\usepackage{amssymb}

\newcommand{\R}{\mathbb{R}}
\theoremstyle{plain}
\newtheorem{theorem}{Theorem}
\newtheorem{lemma}[theorem]{Lemma}

%%%%%%%%%%%%%%%%%%%%%%%%%%%%%%%%%%%%%%%%%%%%%%%%%%%%%%%%%%%%%%%%
% This is FULLPAGE.STY by H.Partl, Version 2 as of 15 Dec 1988.
% Document Style Option to fill the paper just like Plain TeX.

\typeout{Style Option FULLPAGE Version 2 as of 15 Dec 1988}

\topmargin 0pt
\advance \topmargin by -\headheight
\advance \topmargin by -\headsep

\textheight 8.9in

\oddsidemargin 0pt
\evensidemargin \oddsidemargin
\marginparwidth 0.5in

\textwidth 6.5in
%%%%%%%%%%%%%%%%%%%%%%%%%%%%%%%%%%%%%%%%%%%%%%%%%%%%%%%%%%%%%%%%

\pagestyle{empty}
\setlength{\oddsidemargin}{0in}
\setlength{\topmargin}{-0.8in}
\setlength{\textwidth}{6.8in}
\setlength{\textheight}{9.5in}


\def\ind{\hspace*{0.3in}}
\def\gap{0.1in}
\def\bigap{0.25in}
\newcommand{\Xomit}[1]{}


\begin{document}
For \textit{degree} $l \geq 0$ and \textit{order} $-l \leq m \leq l$, we have \textit{spherical harmonics}
\[
Y^m_l(\lambda, \theta) := 
\begin{cases}
\sqrt{2}a^m_l P^m_l(cos \theta)cos(m\lambda) & m > 0\\
a^0_l P_l(cos \theta) & m = 0\\
\sqrt{2}a^{|m|}_l P^{|m|}_l(cos \theta)sin(m\lambda) & m < 0\\
\end{cases}
\]
\begin{description}
\item[$\bullet$] where $P^k_l$, for $0 \leq k \leq l$ is the associated Legendre function

\item[$\bullet$] where $a^k_l$ is the normalization factor:
\[
a^k_l := \sqrt{\frac{(2l+1)(l - k)!}{4\pi(l+k)!}}
\]
\end{description}

Then for $f \in L^2(S^2)$, we have
\[
f(\lambda, \theta) = \sum_{l = 0}^\infty \sum_{m = -l}^l c^m_l Y^m_l(\lambda, \theta)
\]
where our coefficients $c^m_l$ can be defined as:
\[
c^m_l = \int_{S^2}f Y^m_l dS = \int_0^{2\pi} \int_0^\pi f(\lambda, \theta)Y^m_l(\lambda, \theta)d\theta d\lambda
\]
In fact, this holds for any $n \geq 3$, where for a function $f \in L^2(S^n)$:
\[
f(\theta, \phi) = \sum_{l = 0}^\infty \sum_{m = -l}^l f_{l, m} Y_{l, m}(\theta, \phi)
\]
where our coefficients $f_{l, m}$ are calculated by:
\[
f_{l, m} = \int_0^{2\pi} \int_0^{\pi}f(\theta, \phi)Y^*_{l, m}(\theta, \phi)sin(\phi)d\theta d\phi
\]
[\textit{Note:}]
\[
Y^*_{l, m}(\theta, \phi) = a^m_l P^m_l(cos \phi)e^{-m\phi}
\]
In {\textbf{Zonal Spherical Harmonics}}, we say that:
\[
Z^{(l)}(\theta, \phi) := P_l(cos \theta)
\]
where again we have $P^l$ a Legendre polynomial of degree $l$. We can then say we have general zonal harmonic with fixed axis $x$ and variable $y$ given by $Z^{(l)}_x(y)$.

Thus for any $Y \in H_l$, a finite dimensional Hilbert space of spherical harmonics degree $l$:
\[
Y(x) = \int_{S^{n-1}} Z^{(l)}_x(y)Y(y)d\Omega(y)
\]
\newpage
Finally, we provide an alternate view of our spherical harmonics: again we define our harmonics as:
\[
Y^m_l(\theta, \varphi) = N^m_l P^m_l(cos \theta)e^{im\varphi}
\]
for normalization factor
\[
N^m_l := \sqrt{\frac{(2l + 1)}{4\pi}\frac{(l - |m|)!}{(l + |m|)!}}
\]
and associated Legendre polynomial $P^m_l$.\\

Now define $L^2(S^n)$ as the space of (real) square-integrable functions on the sphere $S^n$. Then, defining
\[
\langle f, g \rangle = \int_{S^n}fg\Omega_n
\]
means that $L^2(S^n)$ is indeed a Hilbert space. In fact, we have 
\[
L^2(S^n) = \bigoplus_{k=0}^\infty \mathcal{H}_k(S^n)
\]
and so for every $f \in L^2(S^n)$, we know that
\[
f = \sum_{l = 0}^{\infty} \sum_{m = -l}^l c_{l, m} Y^m_l
\]
where we can find $c_{l, m}$ by:
\[
c_{l, m} = \langle f, Y^m_l \rangle
\]
\end{document}